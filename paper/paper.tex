\documentclass[conference]{IEEEtran}
\usepackage{graphicx}
\usepackage{amsmath}
\usepackage{cite}
\usepackage{hyperref}

\title{Performance Evaluation of Lightweight Cryptographic Algorithms for IoT Devices}

\author{
\IEEEauthorblockN{Author Name}
\IEEEauthorblockA{
\textit{Department} \\
\textit{Institution} \\
Email: author@example.com
}
}

\begin{document}

\maketitle

\begin{abstract}
This paper presents a comprehensive performance evaluation of lightweight cryptographic algorithms suitable for resource-constrained IoT devices. We benchmark ASCON, AES-CCM, and ChaCha20 algorithms, analyzing their execution time, memory footprint, and energy consumption. Our results provide insights into selecting appropriate cryptographic solutions for different IoT application scenarios.
\end{abstract}

\begin{IEEEkeywords}
Internet of Things, Lightweight Cryptography, ASCON, AES-CCM, ChaCha20, Performance Evaluation
\end{IEEEkeywords}

\section{Introduction}
The proliferation of Internet of Things (IoT) devices has created new security challenges. Traditional cryptographic algorithms may be too resource-intensive for constrained devices. This research evaluates lightweight alternatives that provide adequate security while minimizing computational overhead.

\section{Background}

\subsection{ASCON}
ASCON is a family of authenticated encryption and hashing algorithms designed for lightweight applications. It was selected as the winner of the CAESAR competition for lightweight applications.

\subsection{AES-CCM}
AES in Counter with CBC-MAC (CCM) mode combines counter mode encryption with CBC-MAC authentication, providing both confidentiality and authenticity.

\subsection{ChaCha20}
ChaCha20 is a stream cipher designed as an alternative to Salsa20, offering better diffusion and security properties while maintaining high performance on software platforms.

\section{Methodology}

\subsection{Experimental Setup}
\begin{itemize}
\item Hardware platform specifications
\item Software environment and compiler settings
\item Test data characteristics
\end{itemize}

\subsection{Benchmarking Approach}
We measure:
\begin{itemize}
\item Encryption/decryption time
\item Memory usage
\item Code size
\item Energy consumption
\end{itemize}

\section{Results}

\subsection{Performance Comparison}
% Include graphs and tables here
% \includegraphics[width=\columnwidth]{../results/graphs/performance_comparison.png}

\subsection{Analysis}
Discussion of the results and their implications for IoT device security.

\section{Conclusion}
Summary of findings and recommendations for selecting cryptographic algorithms for specific IoT use cases.

\bibliographystyle{IEEEtran}
\bibliography{references}

\end{document}
